\documentclass[14pt]{extarticle}

\usepackage{polyglossia}
\setmainlanguage{ukrainian} 
\setotherlanguage{english}
\addto\captionsukrainian{\renewcommand{\contentsname}{ЗМІСТ}}

\newfontfamily\ukrainianfont[Script=Cyrillic,Ligatures=TeX]{Times New Roman}
\newfontfamily\englishfont[Script=Latin,Ligatures=TeX]{Times New Roman}

% first section number
%\setcounter{section}{0}

% dotted line for table of contents vanilla
%\makeatletter
%\renewcommand*\l@section{\@dottedtocline{1}{0em}{1em}}
%\makeatother

\usepackage{titlesec}
\titleformat{\section}[block]{\normalfont\filcenter}{}{0em}{\thesection. }{}
\titleformat{\subsection}[block]{\normalfont\filcenter}{}{0em}{\thesubsection. }{}
\titlespacing{\section}{0pt}{0pt}{0pt}
\titlespacing{\subsection}{0pt}{0pt}{0pt}

\usepackage{tocloft}
\renewcommand\cftsecfont{\normalfont}
\renewcommand\cftsecpagefont{\normalfont}
\renewcommand{\cftsecleader}{\cftdotfill{\cftdotsep}}
\renewcommand{\cftsecaftersnum}{.}%
\renewcommand{\cfttoctitlefont}{\hspace*{\fill}}
\renewcommand{\cftaftertoctitle}{\hspace*{\fill}}

\makeatletter
\renewcommand\@biblabel[1]{#1.}
\makeatother

\usepackage{cite}

\begin{document}

\tableofcontents

\vspace{1in}

\section{РОЗДІЛ 1}
\subsection{Пункт 1}
текст
\subsection{Пункт 2}
текст
\section{РОЗДІЛ 2}
\subsection{Пункт 1}
посилання \cite{bib1,bib3,bib4,bib5}
\subsection{Пункт 2}
текст

\begin{thebibliography}{9}
\bibitem{bib1} 
Посилання 1
\bibitem{bib2} 
Посилання 2
\bibitem{bib3} 
Посилання на
\begin{otherlanguage}{english}%
"English book"%
\end{otherlanguage}
3

\bibitem{bib4}
Посилання 4

\bibitem{bib5}
Посилання 5

\end{thebibliography}

\end{document}